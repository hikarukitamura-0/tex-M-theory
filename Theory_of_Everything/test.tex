% !TEX program = lualatex 
%   latexmk -lualatex test.tex

\documentclass{ltjsarticle}
\usepackage{amsmath, amssymb} % 数学用パッケージ
\usepackage{graphicx} % 図を入れる用
\usepackage{hyperref} % 目次やリンク

\begin{document}

\title{M理論からLorentz力の式を導く}
\author{北村 光瑠}
\maketitle


\section{はじめに・物理学の願い}
物理学というものは, 高校まででは物質の運動や現象について, 実験的に求められた式を公式として覚えるだけという無味乾燥な学問になっている. 
だが, 本来物理学というのはただ物質について式を並べているだけでなく, 一つの壮大な願いをかなえるための学問である. 
その願いとは, すべての力と物質を一つの理論で説明したいというものである. 
その一例として万物の理論(Theory of Everything; ToE)というものが存在する. 
この理論は名前からもわかるように, 世界で起こるすべての物理現象を一つの理論で説明できるような理論であるが, 当然現在も研究中だ. 
だが, 物理学を進めていくにつれ, 扱うスケールが小さくなってゆく. 
Newtonがリンゴを見て万有引力の法則を閃いたのは有名な話だが, 物事の根本を論じるには
リンゴを超え, 物質を構成する原子にたどり着き, さらにその原子の中身までに迫らなければならないのだ($10^{-15}~\mathrm{m}$, Fmスケールである). 
しかし, そんな考えられないような小さな世界でも, 確かに我々の住む宇宙と深く繋がっている. なぜなら宇宙もまた最小単位の集まりで構成されているからだ. 
今回は, このようなミクロな世界を論じる新理論「M理論(M-Theory)」から我々がよく目にするLorentz力の式を導いて, その強い繋がりを証明しようと思う. 


\section{素粒子基礎}
\subsection{基本相互作用}
まず, 万物の理論などの物理理論を学ぶ前に, それら理論の舞台となる素粒子について説明する. 
物理学には, 電磁気学や量子力学, 古典力学など様々な分野がある. 
それらの現象はすべて素粒子の基本相互作用(Fundamental interaction)によって引き起こされていると考えられており, フェルミ粒子がゲージ粒子を交換することで力が伝達される(素粒子の種類については次セクションで説明する). 
基本相互作用とは物質の最小単位である素粒子の間で相互に作用する力のことで, 「電磁相互作用(Electromagnetic interaction)」「弱い相互作用(Weak interaction)」
「強い相互作用(Strong interaction)」「重力相互作用(Gravitational interaction)」がある. 
これらの中でも特に「電磁相互作用」「弱い相互作用」「強い相互作用」の三つを記述するためのモデルを標準模型(Standerd Model; SM)と呼ぶ. 

\subsection{素粒子の種類}
現在見つかっている素粒子は17個で, 標準模型という化学でいう周期表のようなもので分類される. 
物質を形作る12種類のフェルミ粒子, その粒子たちが相互作用するときに橋渡しとなる(相互作用の力の媒介)5種類のボース粒子がある. 
フェルミ粒子は6種類のクオーク, 6種類のレプトンと2種類に分類でき, クオークは強い相互作用を受けるが, レプトンは受けないという違いがある. 
さらに, クオークは世代というものが存在し, それぞれ3世代ある. 
これは3世代以上のクオークが存在すると, CP対称性の破れ
\footnote{
    CP対称性とは「粒子を反粒子にし, かつ左右を入れ替えても物理法則は同じ」という対称性だ. 
バリオン非対称性(宇宙に物質と反物質が等しく存在しない)の起源の一つではないかと考えられている. 
}
を説明することができるというKobayashi-Masukawa理論(2008年ノーベル物理学賞受賞)から生まれたものである. 

\begin{table}[h]
\centering
\begin{tabular}{c|c|c|c|p{6cm}}
相互作用 & 強さ & ゲージ粒子 & ゲージ対称性 & 概要 \\
\hline
電磁相互作用 & $10^{-2}$ & 光子(フォトン) & $U(1)$ & 原子間共有結合やCoulomb力などに作用する. 強い力より弱いため, 原子核がCoulomb力の斥力によって崩壊することはない.  \\
弱い相互作用 & $10^{-13}$ & $W^{\pm}$/$Z^0$ボソン & $SU(2)$ & 放射性崩壊や中性子→陽子変換などに関わる.  \\
強い相互作用 & $1$ & グルーオン & $SU(3)$ & 原子核の中の陽子・中性子を結びつける力. 距離が離れるほど強くなる(カラー閉じ込め).  \\
重力相互作用 & $10^{-38}$ & グラビトン(仮説) & 不明 & 質量を持つもの同士に働く力. 非常に弱いため素粒子レベルでは無視できる.  \\
\end{tabular}
\caption{基本相互作用の比較}
\label{tab:forces}
\end{table}



%\subsection{ゲージ対称性}


%\subsection{標準模型}
%この標準模型は, 多くの物理現象をほぼ的確に記述することができるが, 基本相互作用の中に含まれている重力相互作用を説明することはできない. 







\section{万物の理論と理論候補}
\subsection{万物の理論}
先述した通り, 物理学の目標として万物の理論というものが存在する. 
「すべての力と物質を一つの理論で説明」では少々曖昧なので, もう少し具体的に説明する. 
先述した通り, 標準模型では電磁・弱・強の三つの基本相互作用を統一的に説明できる. 
万物の理論とは, 標準模型では説明できない重力をも含め四つの基本相互作用を統一的に説明する理論のことである.   
すなわち, 電磁・弱・強・重力のすべてをまとめて記述できる理論を指す. 


\subsection{弦理論・超弦理論・M理論の関係}
万物の理論の候補として様々な理論が議論されているが, その中でも特に有力なのが超弦理論である. 
そもそも弦理論とは粒子の相互作用を記述する際に, 




\section{古典力学から量子力学へ}
古典力学とは量子力学が登場する以前の力学のことを指す. 
ニュートン力学とも呼ばれる. 
もちろん古典力学が間違っていたので, 量子力学ができたのではなく, ある条件下だとニュートン力学が成立しない場面が存在するのである. 
ではなぜそのように力学が分裂してしまったのか. 

原子の構造を説明するためのモデルとしてボーアモデルというものがある. 
しかし, このボーアモデルは定常状態という特殊な状態を認めなければ成立しない. 
ボーアモデルでは, 電子は原子核の周りを周回しているとしている. すなわち電子が角運動量を持っているということだ. 
だが, Maxwell方程式によると, 「電荷が加速すると電磁波を放出する」 と定められている. 
ここで一つの矛盾が発生する. 電子は電荷をもっており, 尚且つ角運動量を持っているので加速もしている. 
これでは, 電子のエネルギーを電磁波として放出してしまい, 電子が運動を止めてしまう. 
さらに原子は中性子と陽子で構成される原子核と, 負の電荷をもつ電子で構成されてるため, 
電磁相互作用のCoulomb力により電子と陽子間に引力が発生する. 
これでは運動エネルギーを失った電子が, Coulomb力により電子が原子に墜落し, 原子が崩壊してしまう. 
ここで登場するのが, 量子力学における電子の確率解釈である. 

量子力学では, 波動関数(Wave function)によって電子を確率の分布(存在の揺らぎ)として表現し, この矛盾を解消させる. 
物質を観測するには電磁波(光子)を衝突させ, 相互作用を発生させることにより観測することができる. 
我々が物質を見るということも, 光子が物質に衝突し, 可視光線を反射して目に入射するから観測できるのである. 
しかし, 観測したい物質が電磁波の波長より小さい場合, 物質を電磁波がすり抜けてしまい観測することができない. 
よってさらに波長を小さくする必要がある. 
仮に電子を電磁波によって観測する場合, 
電子の大きさを
$r_e<10^{-19}~\mathrm{m}$, 
換算Planck定数を
$\hbar = 1.054571817×10^{−34}~\mathrm{J}$, 
光速を
$c=299,792,458~\mathrm{m/s}$
と仮定すると
(現在の理論では半径は0としか言えないが, 現在の実験精度的にはこれが上限とされている)
Planck-Einsteinの関係式
\[
E = \hbar\nu \quad \nu = \frac{c}{\lambda}
\]
ここで
\[
\lambda < 10^{-19}~\mathrm{m}
\]
各値を代入すると
\[
\nu > \frac{c}{\lambda} > \frac{2.99792458 \times 10^8}{10^{-19}} = 2.99792458 \times 10^{27}~\mathrm{Hz},
\]

\[
E > \hbar \nu = (1.054571817 \times 10^{-34}) \times (2.99792458 \times 10^{27}) \approx 3.16 \times 10^{-7}~\mathrm{J},
\]

\[
E > \frac{3.16 \times 10^{-7}}{1.602 \times 10^{-19}} \approx 1.97 \times 10^{12}~\mathrm{eV} \approx 2~\mathrm{TeV}.
\]
したがって, 電子の大きさが $10^{-19}~\mathrm{m}$ 以下であれば, 対応する光子のエネルギーは
\[
E \gtrsim 2~\mathrm{TeV}
\]
となる. 
しかし, これほどまでに巨大なエネルギーを持つ光子を電子に衝突させると, 電子に大きな運動量を与えてしまい, 運動量がわからなくなる. 
これが, 位置と運動量を同時に正確には測れないということを示すHeisenbergの不確定性原理
\[
\Delta x \, \Delta p \gtrsim \frac{\hbar}{2}
\]
である. 
ここで, $\Delta x$ は位置の不確定性, $\Delta p$ は運動量の不確定性, $\hbar$ は換算Planck定数である. 

\section{量子力学と相対性理論}
\subsection{特殊相対性理論}
まず相対性理論(Theory of relativity)は, 1905年にEinsteinによって発表された特殊相対性理論(Special relativity)と, 
その後同じくEinsteinが1915年に発表した一般相対性理論(General Theory of Relativity; GR)の二つに分類できる. 
二つの理論の違いは, 理論の中に重力を含むか含まないかである. 
この二つの中でもまずは重力を含まない特殊相対性理論について解説する. 

特殊相対性理論は重要な二つの原理を土台として成り立っている. 
それは「特殊相対性原理」「光速度不変の原理」である. 
これらの原理により, 特殊相対性理論では
\begin{itemize}
    \item $u_x$ 動くものの時間は遅れる
    \item $u'_x$ 動くものの長さは遅れる
    \item $v$ 「エネルギー」=「質量」
\end{itemize}
という特徴が表れる. 

\subsubsection{特殊相対性原理}
特殊相対性原理とは「どの慣性系(静止している座標系か等速直線運動している座標系)から見たとしても, 物理法則は不燃である」という原理である. 
つまり「静止している」「等速運動している」ということはあくまで相対的でしかないため, 
止まっている観測者から見ても, 一定の速さで動いている観測者から見ても, 物理法則(たとえばNewtonの運動法則やMaxwell方程式)は同じように成り立つ. 
地球上で物体を落下させれば, 通常の物理法則と合致するが, それは等速運動するような座標系でも同じである. 
例として, 電車が一定の速さでまっすぐ走っているとき, その中でボールを投げても, 地上に静止しているときと同じように運動方程式が成り立つ. 

\subsubsection{光速度不変の原理}
光速度不変の原理とは「どの慣性系から見ても光の速さは変わらない」というものだ. 
例えば, 静止しているAと, $x$方向に$v~\mathbf{m/s}$で移動しているBから, $x$方向に$c~\mathbf{m/s}$で動く光cを考える. 
もちろんAから見たcの速度は$c~\mathbf{m/s}$であるが, Bからcの速度を見たとしても$v + c ~\mathbf{m/s}$とはならず, $c~\mathbf{m/s}$と見える. 
こうした光速度が一定に保たれてしまうような不思議な原理をまずは認めなければならない. この原理は実験により証明されている. 
この原理を成り立たせるために特殊相対性理論ではLorentz変換を利用し, 理論体系を構築している. 
Lorentz変換を知るために, まずは座標変換について学ばなければならない. 

\subsection{Galilei変換とLorentz変換}
そもそも座標変換とは何か. 
一番直感的で身近なのは古典力学における大前提「Galilei変換」である. 
ある座標系 \(S'\) が速度 \(\mathbf{v} = (v_x, v_y, v_z)\) で移動しているとき, 
位置と時間のGalilei変換は

\[
\begin{cases}
x' = x - v_x t \\

y' = y - v_y t \\
z' = z - v_z t \\
t' = t
\end{cases}
\]

ここで
- \((x, y, z, t)\) : 静止系 \(S\) の座標  
- \((x', y', z', t')\) : 移動系 \(S'\) の座標  
- 時間は絶対 (\(t' = t\))  (どの座標系のどのような位置でも時間は同じであるということ)

これを具体例で考えてみる. 
$x$軸方向に速度$15\mathbf{m/s}$で動く電車がある. 
その車内でBがボールを$x$軸方向に$5\mathbb{m/s}$で投げる. 
電車の外で静止しているAからボールを観測した場合, ボールの速度は

\[
15 + 5 = 20\mathbf{m/s}
\]

となる. 
この結果を, 先ほど示したGalilean変換式を用いて表現してみる. 
$x$軸方向に速度$v = 15\mathbf{m/s}$で動く電車を座標系$T'$とし, 静止しているAを座標系$T$とすると位置と時間のGalilei変換は

\[
\begin{cases}
x' = x - vt = x - 15 t \\
y' = y \\
z' = z \\
t' = t
\end{cases}
\]
速度は位置を時間で微分すればよいので(どのくらいの時間でどれだけ変化したか)
\[
\begin{aligned}
u_x &= \frac{dx}{dt} = \frac{d}{dt}(x' + v t) = \frac{dx'}{dt} + \frac{d(vt)}{dt} = u'_x + v = 15 + 5 = 20\mathbf{m/s}
\end{aligned}
\]
ここで, $u_x$ は地上の観測者から見たボールの速度, $u'_x$ は電車内の観測者から見たボールの速度, $v$ は電車の速度である. 

これは, 先ほどの加算式と完全に一致する. 
このように異なる座標系(Aから見たボールの見え方と, ボールから見たAの見えかたは違う→座標系が違う )間で, 
相対速度を計算し, 別の座標系での表現に変換することを座標変換という. 
特に, この「どの座標系のどの位置でも時間は同じ(絶対的である)」というような座標変換がGalilei変換である. 

しかし, この座標変換では光速度不変の原理を成り立たせることはできないのである. 光速を$c$とおき, 
$x$軸方向に速度$0.7c\mathbf{m/s}$で動く電車を座標系$T'$とし, 車内にいるBが$x$軸方向に$0.5c\mathbf{m/s}$でボールを投げる. 
静止しているAを座標系$T$とすると$T$と$T'$における, 位置と時間のGalilei変換は
\[
0.5c + 0.7c = 1.2c
\]
よって光速は常に$c$であるという光速度不変の原理が成り立たなくなってしまう. 
そこで今度は, 光速度cが常に一定になるような変換を考えてみる. 
原点から光が放たれる場合を考えてみよう
特殊相対性理論におけるLorentz 変換は,光速度 $c$ を一定とするように座標系の変換を与えるものです.

$x$ 軸方向に速度 $v$ で運動する慣性系 $T'$ と,静止している慣性系 $T$ の間での座標変換は

\[
\begin{cases}
x' = \gamma (x - v t) \\
y' = y \\
z' = z \\
t' = \gamma \left( t - \dfrac{v}{c^2} x \right)
\end{cases}
\]

となります.

ここで
\[
\gamma = \frac{1}{\sqrt{1 - \dfrac{v^2}{c^2}}}
\]
をローレンツ因子と呼びます.

---

### 速度の変換則
Lorentz 変換を用いた速度変換は次のようになります:

\[
u'_x = \frac{u_x - v}{1 - \dfrac{u_x v}{c^2}}, 
\qquad
u'_y = \frac{u_y}{\gamma \left(1 - \dfrac{u_x v}{c^2}\right)},
\qquad
u'_z = \frac{u_z}{\gamma \left(1 - \dfrac{u_x v}{c^2}\right)}.
\]

光速度 $c$ はどの座標系においても不変($u = c$ なら $u' = c$)であることが保証されています.




\section{ゲージ対称性による基本相互作用の記述}
\section{弦理論, 超弦理論からM理論へ}
\section{M理論からLorentz力の式へ}
\section{結論と展望}

\section{テスト用}
\subsection{小セクション}
\subsubsection{さらに小さいセクション}
\paragraph{段落見出し} テキスト...

\begin{figure}[h]
\centering
\caption{実験装置の模式図}
\label{fig:setup}
\end{figure}

図 \ref{fig:setup} に示すように…
表 \ref{tab:result} を参照。


\begin{itemize}
    \item $u_x$ は地上の観測者から見たボールの速度
    \item $u'_x$ は電車内の観測者から見たボールの速度
    \item $v$ は電車の速度
\end{itemize}

% 文献リスト
\begin{thebibliography}{9}
\bibitem{einstein1905} A. Einstein, Ann. Phys. 17, 891 (1905).
\end{thebibliography}

速度 $v = \frac{dx}{dt}$ は…
\[
E = mc^2
\]

% 複数行は align 環境
\begin{align}
F &= ma \\
E &= \frac{1}{2}mv^2
\end{align}


\end{document}
