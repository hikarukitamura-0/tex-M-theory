\documentclass{ltjsarticle}
\usepackage{titlesec}
\usepackage{amsmath, amssymb} % 数学用パッケージ
\usepackage{graphicx} % 図を入れる用
\usepackage{hyperref} % 目次やリンク
%latexmk -lualatex kyouzai.tex

\begin{document}
\begin{titlepage}
    \centering
    \vspace*{3cm} % 上からの余白を調整
    {\Huge \bfseries 物理数学基礎}\\[2cm] % 題名(フォントサイズ大&太字)
    {\Large 物理学のための数学}\\[1cm]
    {\large 北村光瑠}\\[5cm]
    {\large 2025年8月30日}
    \vfill % 下側に空白を自動調整して配置
\end{titlepage}

\section{集合論基礎}
数学をより厳密に定義するには, 群, 環, 体という抽象代数学の三大構造を知る必要がある. 
さらに, 物理学で利用されている例として, ゲージ対称性のゲージ群なども群で表現される. 
\footnote{
    高校で習うような集合の進化版だと考えてよいが, その進化度合いが桁違いであるほど深い学問である. 
    かつて数学者は群を用いて数学のあらゆる理論を統一しようとした(群にはそれほど強大な力がある). 
    しかし, Gödelの不完全性定理によって, 数学全体を有限の公理系で完全に統一することは不可能であることが示された. 
    現代数学は, それでもなお各分野における理論の統一や体系化を目指し, 発展を続けている. 
}

\subsection{群(Group)}

\textbf{定義:} 集合 $G$ と二項演算 $\cdot: G \times G \to G$ が与えられているとき, 次を満たす場合, $(G, \cdot)$ を\textbf{群}という. 
\begin{enumerate}
    \item 閉性: ∀\footnote{∀: 任意の~に対して  ∀ $a$: 任意のaに対して} $a,b \in G$ , $a \cdot b \in G$
    \item 結合法則: ∀ $a,b,c \in G$ , $(a \cdot b) \cdot c = a \cdot (b \cdot c)$
    \item 単位元の存在: $\exists\footnote{∃: ~が存在する. ∃ $a$: $a$が存在する} e \in G, \forall a \in G, \ e \cdot a = a \cdot e = a$
    \item 逆元の存在: ∀ $a \in G$ ∃ $a^{-1} \in G$ , $a \cdot a^{-1} = a^{-1} \cdot a = e$
\end{enumerate}
さらに, 可換性: ∀ $a,b \in G$ , $a \cdot b = b \cdot a$ である場合, \textbf{可換群 (Abelian group)} という. 

要するに群とは「集合と1つの演算(加法乗法等)」でできたシステムで, 
\begin{enumerate}
\item 二項演算したとしても集合の中に収まる(閉性)
\item 結合法則が成り立つ
\item「何も変えない要素(単位元)\footnote{
単位元とは, 二項演算をしても何も変わらない操作をするもので, 加法群では $1+0=1$ の$0$, 乗法群では $3\times 1=3$ の$1$である. 
乗法群を定義するとき, 単位元が集合の要素に存在しないと乗法群は作れない(集合$\{2, 5, 7\}$で乗法群を作る場合, 
集合に単位元が存在しないため, $3\times1$を定義できない→乗法が成り立たない). 
}」がある
\item「単位元に戻せる操作(逆元)\footnote{
逆元とは, 二項演算をして逆元にする操作をするもので, 加法群では $1+(-1)=0$ の$-1$, 乗法群では $3\times3^{-1}=1$ の$3^{-1}$である. 
集合\{2, 3, 5\}で乗法群を考える. (単位元1は含めない)$2x=3$という方程式を解くとき, 逆元$2^{-1}$が存在しないため方程式を解くことができない. 
}」がある
\end{enumerate}
という4つの条件が揃ったものである. 
群の中で定義されるのは二項演算だけだが, 
二項演算を一つの項として扱うことによって問題なく多項演算が可能である(特別に多項演算を定義する必要はない).

\subsection{環(Ring)}

\textbf{定義:} 集合 $R$ と二項演算 $+(加法): R \times R \to R$, $\cdot(乗法): R \times R \to R$ 
が与えられているとき, $(R,+,\cdot)$ が次を満たす場合、\textbf{環}という. 
\begin{enumerate}
    \item $(R,+)$が可換群
    \item $(R,\cdot)$が結合法則を満たす半群
    \item 分配法則: ∀ $a,b,c \in R$ , $a\cdot(b+c) = a\cdot b + a \cdot c, \quad (a+b)\cdot c = a\cdot c + b\cdot c$

\end{enumerate}

さらに, $(R,\cdot)$が可換半群(半群で可変性$a・b=b・a$が成り立つ)である場合は, 
\textbf{可換環(commutative ring)}という. 
ここで半群とは, 結合法則, 閉性が成り立つが, 単位元や逆元は要求されないような群である. 

環は, 一つの演算しか扱えない群とは違い「加法, 乗法の二演算」がある世界である. 
乗法は逆元があるとは限らないが, 分配法則によって加法と乗法がうまく共存している
(整数$\mathbb{Z}$は環であるが, 加法乗法が成立している). 

\subsection{体(Field)}

\textbf{定義:} 可換環 $(F,+,\cdot)$ が次を満たすとき、\textbf{体}という. 
\begin{enumerate}
    \item $(F,+)$が可換群
    \item $0 \neq a \in F$ , $∃ a^{-1} \in F \text{ such that} 
    \footnote{such that : 「~となるような」 / 「~という条件で」 
    $0 \neq a \in F , a^{-1} \in F が存在する条件として a\cdot a^{-1} = a^{-1}\cdot a=1 が成り立つ$} 
    a\cdot a^{-1} = a^{-1}\cdot a=1$
\end{enumerate}

体は分数まで含めた全ての演算が可能な世界で, 四則演算(0除算を除く)がすべて可能な完全な代数構造である. 
有理数$\mathbb{Q}$, 実数$\mathbb{R}$, 複素数$\mathbb{C}$などが含まれる. 

\section{線形代数}
線形代数は数学および物理学における基礎理論であり, ベクトル, 行列, 内積といった概念は
物理学では必須である. 特に現代の科学技術においては, 
抽象的なベクトル空間の枠組みが必須となっている. 本稿では, 線形代数の基礎的内容である
ベクトル, 基底, 内積, 双対空間, 線形写像, そして基底変換を詳細に議論する. 

\subsection{ベクトルの概念}
「ベクトル」という語は高校数学や物理でよく登場し, 平面や空間における矢印として直感的に理解される. 
しかし, 線形代数ではベクトルはもっと一般化された抽象的対象である. 

高校まででは, ベクトルを二次元や三次元の座標系でしか表さなかった. 
だが, より一般的な考え方として, 抽象ベクトルというベクトルが存在し, それを二次元や三次元の座標系で選ぶことで
抽象ベクトルが意味を持つようになると考える. 

\subsubsection{抽象ベクトル空間}
より一般に, 体 $\mathbb{K}$(通常は $\mathbb{R}$ や $\mathbb{C}$)上の集合 $V$ がベクトル空間であるとは, 
加法 $+:V\times V \to V$ とスカラー倍 $\cdot : \mathbb{K}\times V \to V$ が定義され, 次の公理を満たすときである:
\begin{enumerate}
    \item $(V,+)$ が可換群
    \item ∀ $a \in \mathbb{K} \ v,w \in V,  a(v+w)=av+aw$
    \item ∀ $a,b \in \mathbb{K} \ v \in V,(a+b)v=av+bv$
    \item ∀ $a,b \in \mathbb{K} \ v \in V,a(bv)=(ab)v$
    \item 単位的性質 ∀ $v \in V , 1v=v$
\end{enumerate}
ここで「抽象的」というのは, 成分や矢印としての表現を持たずとも構造として定義できることを意味する
(どの方向, どの長坂は座標系に依存しない事実だから). 

抽象ベクトルだけでは意味を持った量で表現することができない. 
イメージは適当な空間に矢印(ベクトル)だけが浮かんでいるような感じだ. 
矢印そのものは変化しないが, その矢印を意味のある量で表すにはものさしで測るしかない. 
だが, そのものさし(基底)の種類によってその矢印の大きさが異なるように見える. 

\subsubsection{線形独立・線形従属}
基底の条件において, 線形独立という用語が登場する. 
線形独立であるベクトルの数によってその空間の次元が決定するため, 線形代数では非常に重要な概念の一つである. 

体 $\mathbb{K}$ 上のベクトル空間 $V$ において, 
ベクトル $v_1, v_2, \dots, v_n \in V$ が
\textbf{線形独立}であるとは,
\[
a_1 v_1 + a_2 v_2 + \cdots + a_n v_n = 0
\]
を満たす $a_1, a_2, \dots, a_n \in \mathbb{K}$ が
$a_1 = a_2 = \cdots = a_n = 0$
のときに限ることである. 
逆に, 0 でない解が存在するならば, 
これらのベクトルは \textbf{線形従属}である. 

線形従属の例として以下のような式がある. 
\[
v_1 = (1,0) , v_2 = (2,0) , \quad a_1 v_1 + a_2 v_2 = 0
\]

この場合, $a_1=a_2=0$以外の解「$a_1=2, a_2=-1$」でも式が成立し, このベクトルでは0以外の解が存在することがわかる. 
この結果の数学的意味として, 二つのベクトルの各方向の成分が重複してしまっている(方向がかぶっている)ことがわかる. 
だが, 線形独立の場合はそれぞれが「独立した」方向情報を持っているため, 
重ね合わせても打ち消せる関係がない. 
ゆえに零ベクトル(全ての成分が0のベクトル)を創る唯一の方向は, 
すべてのベクトルをゼロでスカラー倍することしかなくなってしまう. 
空間の構造を無駄なく記述することができる. 

特に, 線形独立なベクトルの集合の大きさ(ベクトルの本数)が
その空間の \textbf{次元} を与える. 
また, 空間を生成する最小の線形独立な集合を \textbf{基底} と呼び, 
任意のベクトルはこの基底の一次結合として一意的に表すことができる. 

例えば, 平面 $\mathbb{R}^2$ では, 
$v_1=(1,0), v_2=(0,1)$ が基底をなし, 
任意のベクトル $(x,y) \in \mathbb{R}^2$ は 
\[
(x,y) = x \cdot v_1 + y \cdot v_2
\]
と一意に表すことができる. 

一方で, $v_1=(1,0), v_2=(2,0)$ のように
線形従属なベクトルの組では, 
空間全体を記述することはできず, 
独立した方向情報が不足しているため基底とはならない. 
このように, 線形独立という性質は,
ベクトル空間の基礎を形づくる上で欠かせない概念である.

\subsubsection{基底の定義}
基底の定義を詳しく見てみよう. 
集合 $\{e_1,\dots,e_n\}$ が $V$ の基底であるとは, 
\begin{enumerate}
    \item 線形独立である. 
    \item 任意の $v \in V$ が $v=\sum_{i=1}^n v^i e_i$ と表される. 
\end{enumerate}
このとき係数 $v^i$ を $v$ の座標(成分)という. 

さらに, 一つの成分が1で, それ以外の成分が0であるような基底を標準基底という. 
$\mathbb{R}^3$ における標準基底は
\[
\mathbf{e_1}=(1,0,0)^\mathrm{T}, \quad \mathbf{e_2}=(0,1,0)^\mathrm{T}, \quad \mathbf{e_3}=(0,0,1)^\mathrm{T}
\]
である. 任意の抽象ベクトル$\mathbf{v}=(x,y,z)$ はこの標準基底によって $v=x e_1+y e_2+z e_3$ と展開される. 
これを\textbf{基底ベクトルによる展開}という. 

基底は座標系において一意的ではなく, 別の基底を選んでもよい. 
例えば $\mathbb{R}^2$ では $(1,1),(1,-1)$ も基底となる. 
ベクトルそのものは基底に依存しないが、座標は基底に依存する. 

例えば同じベクトルだったとしても, そのベクトルが基底$e_i$に存在する場合と, 
別の異なる基底$e'_i$で存在する場合には, 表し方が異なる(ベクトル自体が変化しているわけではない). 

\subsubsection{具体的ベクトル}
では基底によって展開することができたベクトルについて議論しよう. 
ユークリッド空間
\footnote{
    ユークリッド空間とは, 我々がこれまで考えてきた普通の幾何学空間の名前をより厳密に書いたものである. 
    この普通の空間の定義は, 紀元前3世紀ごろに数学者ユークリッドが編纂したと言われる数学書から導かれる. 
    一般相対性理論では, 重力を時空の歪み, 即ち「空間の歪み」として解釈するため, 曲がった空間である非ユークリット空間が登場する. 
}$\mathbb{R}^n$
\footnote{
    $\mathbb{R}^n$ というのは実数に$n$方向があることを表している. 
    例えば$\mathbb{R}^3$だと, 三方向がある実数, 即ち三次元空間を意味する. 
} は $n$ 個の実数を成分に持つ数列で表される:
\[
\mathbf{v} = (v^1, v^2, \dots, v^n)
\]
このとき加法とスカラー倍は次のように定義される:
\[
(v^1,\dots,v^n)+(w^1,\dots,w^n)=(v^1+w^1,\dots,v^n+w^n)
\]
\[
a(v^1,\dots,v^n)=(av^1,\dots,av^n)
\]

ベクトルの書き方として, 行ベクトルは上に添字をつけ, 列ベクトルは下に添字をつける. 
行ベクトル$\mathbf{v^{i}}$, 列ベクトル$\mathbf{v_j}$を以下に示す. 
なお, 列ベクトルを転置表記 $^\mathrm{T}$を付けた行ベクトルのように表すこともある. 
\[
\mathbf{v^{i}} =
\begin{pmatrix}
v_1 , v_2 , \cdots , v_i
\end{pmatrix},
\quad
\mathbf{v_j} =
\begin{pmatrix}
v_1 \\
v_2 \\
\vdots \\
v_j
\end{pmatrix},
\quad
\mathbf{v_j} = (v_1 , v_2 , \cdots , v_j)^\mathrm{T}
\]

\subsection{正規直交基底}
\subsubsection{内積の導入}
ユークリッド空間$\mathbb{R}^n$の場合, $\mathbf{v}, \mathbf{w}$における内積とは
\[
\langle \mathbf{v}, \mathbf{w} \rangle \footnote{内積は<このような特殊なカッコで示す>} = \sum_{i=1}^n v^i w^i
\]

例えば, $\mathbf{v}=(1,2,3)$, $\mathbf{w}=(4,5,6)$ の場合:
\[
\langle \mathbf{v}, \mathbf{w} \rangle = 1 \cdot 4 + 2 \cdot 5 + 3 \cdot 6 = 32.
\]

内積から導かれる概念として以下のようなものがある. 
\begin{itemize}
    \item \textbf{ノルム(長さ)}:$\|\mathbf{v}\| = \sqrt{\langle \mathbf{v}, \mathbf{v} \rangle}$
    \item \textbf{角度}:$\cos \theta = \dfrac{\langle \mathbf{v}, \mathbf{w} \rangle}{\|\mathbf{v}\| \|\mathbf{w}\|}$
    \item \textbf{直交}:$\langle \mathbf{v}, \mathbf{w} \rangle = 0 \iff \mathbf{v} \perp \mathbf{w}$
\end{itemize}

\subsubsection{定義}
基底 $\{e_1,\dots,e_n\}$ が正規直交基底であるとは、
\[
\langle e_i, e^{j} \rangle = \delta^{j}_i
\]
を満たすことをいう. ここで $\delta_{ij}$ はクロネッカーのデルタであり, $i=j$ のとき1, それ以外では0である. 


\subsubsection{性質}
正規直交基底を用いると、ベクトル $v=\sum_i v^i e_i$ のノルムは
\[
\|v\|^2=\langle v,v\rangle = \sum_i |v^i|^2
\]
と表され、座標計算が極めて容易になる。

---

\subsection{内積の定義}
内積はベクトル空間に幾何的概念(長さ、角度)を導入する。

\subsubsection{実ベクトル空間の内積}
実ベクトル空間 $V$ 上の内積は写像 $\langle \cdot,\cdot\rangle:V\times V\to \mathbb{R}$ であり、
\begin{enumerate}
    \item 対称性:$\langle v,w\rangle = \langle w,v\rangle$。
    \item 双線形性:$\langle av_1+bv_2, w\rangle = a\langle v_1,w\rangle + b\langle v_2,w\rangle$。
    \item 正定値性:$\langle v,v\rangle \ge 0$, $=0$ のとき $v=0$。
\end{enumerate}
を満たす。

\subsubsection{例}
ユークリッド空間 $\mathbb{R}^n$ における標準内積は
\[
\langle v,w\rangle = \sum_{i=1}^n v^i w^i
\]
である。

---

\subsection{行列に内積を導入}
\subsubsection{行列空間}
行列空間 $M_n(\mathbb{R})$ もまたベクトル空間であり、加法とスカラー倍は成分ごとに定義される。

\subsubsection{フロベニウス内積}
$A,B\in M_n(\mathbb{R})$ に対し、
\[
\langle A,B\rangle = \mathrm{Tr}(A^TB)
\]
を定めると、これが内積となる。これはフロベニウス内積と呼ばれる。

\subsubsection{例}
$A=\begin{pmatrix}1&2\\3&4\end{pmatrix}, B=\begin{pmatrix}0&1\\-1&2\end{pmatrix}$ のとき、
\[
\langle A,B\rangle = \mathrm{Tr}(A^TB)=1\cdot 0+2\cdot 1+3\cdot (-1)+4\cdot 2=7。
\]

---

\subsection{内積とクロネッカーのデルタ}
正規直交基底 $\{e_i\}$ を用いると、
\[
\langle e_i, e_j\rangle = \delta_{ij}
\]
となり、ベクトルの内積は
\[
\langle v,w\rangle = \sum_{i,j} v^i w^j \langle e_i,e_j\rangle
= \sum_i v^i w^i
\]
と簡潔に表される。この $\delta_{ij}$ が成分の選び出しを担う。

---

\subsection{双対ベクトルと双対空間}
\subsection{定義}
$V$ がベクトル空間のとき、その双対空間 $V^*$ を
\[
V^* = \{ f:V\to \mathbb{R}\mid f \ \text{は線形写像}\}
\]
と定める。$V^*$ の元を双対ベクトル(または共変ベクトル)と呼ぶ。

\subsection{双対基底}
基底 $\{e_i\}$ に対応して双対基底 $\{e^i\}$ を
\[
e^i(e_j)=\delta^i_j
\]
と定める。このとき任意の $f\in V^*$ は $f=\sum_i f_i e^i$ と展開される。

\subsection{内積による同一視}
内積をもつ有限次元空間では、ベクトル $v$ に対応する線形写像 $f_v(w)=\langle v,w\rangle$ を定義できる。
この写像は $V$ から $V^*$ への同型写像を与え、ベクトルと双対ベクトルが自然に同一視される。

---

\subection{ベクトル空間の数学的定義の再確認}
体 $\mathbb{K}$ 上の集合 $V$ がベクトル空間であるとは、
\begin{itemize}
    \item $(V,+)$ が可換群。
    \item $a(v+w)=av+aw$。
    \item $(a+b)v=av+bv$。
    \item $a(bv)=(ab)v$。
    \item $1v=v$。
\end{itemize}
を満たすときである。この枠組みは $\mathbb{R}^n$ のような具体例に限らず、
関数空間や行列空間などにも適用される。

---

\subsection{線形写像}
\subsection{定義}
$V,W$ をベクトル空間とする。写像 $T:V\to W$ が線形写像であるとは、
\[
T(av+bw)=aT(v)+bT(w)
\]
が任意の $a,b\in \mathbb{K}, v,w\in V$ に対して成立することをいう。

\subsection{行列表示}
基底を選ぶと、線形写像は行列で表される。例えば $V=W=\mathbb{R}^2$ で
\[
T(x,y)=(2x+y, x-y)
\]
とすると、基底 $\{e_1=(1,0),e_2=(0,1)\}$ に対して
\[
T(e_1)=(2,1)=2e_1+1e_2,\quad T(e_2)=(1,-1)=1e_1-1e_2
\]
となる。したがって $T$ の行列は
\[
\begin{pmatrix}2&1\\1&-1\end{pmatrix}
\]
である。

---

\subsection{基底変換と座標変換}
\subsection{基底変換}
基底 $\{e_i\}$ から $\{e'_j\}$ への基底変換を考える。
行列 $P=(P_{ij})$ を用いて
\[
e'_j=\sum_i P_{ij} e_i
\]
と表される。

\subsection{座標変換}
ベクトル $v$ を $\{e_i\}$ で $v=\sum_i v^i e_i$ と展開し、また $\{e'_j\}$ で $v=\sum_j v'^j e'_j$ と展開する。
基底変換式を代入すると
\[
v=\sum_j v'^j e'_j = \sum_j v'^j \left(\sum_i P_{ij} e_i\right) = \sum_i \left(\sum_j P_{ij} v'^j\right) e_i.
\]
したがって
\[
v^i=\sum_j P_{ij} v'^j
\]
が成立する。行列式で表すと
\[
\bm{v}=P\bm{v}',
\]
よって
\[
\bm{v}'=P^{-1}\bm{v}
\]
が成り立つ。つまりベクトル自体は変わらないが、成分表示は基底に依存して変換される。

\subsection{例}
$\mathbb{R}^2$ の基底 $e_1=(1,0),e_2=(0,1)$ から $e'_1=(1,1),e'_2=(1,-1)$ へ基底を変える場合を考える。
変換行列は
\[
P=\begin{pmatrix}1&1\\1&-1\end{pmatrix}
\]
であり、$v=(x,y)^T$ の新しい座標は
\[
\bm{v}'=P^{-1}\bm{v}=\frac{1}{2}\begin{pmatrix}1&1\\1&-1\end{pmatrix}(x,y)^T
=\left(\frac{x+y}{2},\frac{x-y}{2}\right)^T
\]
となる。

\section{微分・偏微分}
このsectionでは微分について説明する. 
高校でやったような一変数関数の微分ではなく, 二変数関数を微分するとどうなるかということについて詳細に議論する. 

\subsection{微分基礎}

\subsubsection{微分のいろいろな記法}
まず, 微分の様々な記法について理解しよう. 
同じようなことを別の書き方で書いているだけだが, 時と場合によって使い分けることで, 
式が見やすくなったりする. 今回紹介する記法は以下の三つだ. 
関数 $y = f(x)$ の導関を表記する. 

\subsection*{ライプニッツ記法 (Leibniz notation)}
\[
\frac{dy}{dx}, \quad \frac{d}{dx} f(x)
\]
独立変数 $x$ に関して従属変数 $y$ の微小変化を明示する記法.
高階微分は次のように表される:
\[
\frac{d^2 y}{dx^2}, \quad \frac{d^n y}{dx^n}.
\]

\subsection*{ラグランジュ記法 (Lagrange notation)}
\[
f'(x), \quad f''(x), \quad f^{(n)}(x)
\]
関数 $f$ に対して,ダッシュ記号または括弧付き上付き添字で導関数を表す.

\subsection*{ニュートン記法 (Newton notation)}
\[
\dot{y}, \quad \ddot{y}, \quad \dddot{y}, \quad \cdots
\]
時間 $t$ を変数とする関数 $y(t)$ に対して,時間微分を点で表す.
特に力学でよく用いられる.

例として, 以下のような表記が可能になる. 
\[
\frac{df}{dt} = \frac{df}{dx}\frac{dx}{dt}, \quad \frac{df}{dt} = f'(x)\dot{x}
\]

\subsubsection{偏微分の定義}
多変数関数を微分することを偏微分というが, すべての変数を一気に微分するのではない. 
多変数の中から変数を一つ選びだし, それを微分するのだ(ほかの変数は定数とみなす). 
たくさんの変数があるなかで偏って一つだけを微分しているというようなイメージである. 
ゆえに, そこまで普通の微分と定義も変わらない. 

$f : \mathbb{R}^2 \to \mathbb{R}$ を関数とし,点 $(x_0,y_0) \in \mathbb{R}^2$ を考える.

\subsection*{$x$ に関する偏微分}
もし実数 $L \in \mathbb{R}$ が存在して
\[
\lim_{h \to 0} 
\frac{f(x_0 + h, y_0) - f(x_0,y_0)}{h} = L
\]
となるならば,$f$ は点 $(x_0,y_0)$ において $x$ に関して偏微分可能であるといい,
\[
\frac{\partial f}{\partial x}(x_0,y_0) := L
\]
と定義する.

\subsection*{$y$ に関する偏微分}
同様に,もし実数 $M \in \mathbb{R}$ が存在して
\[
\lim_{h \to 0} 
\frac{f(x_0, y_0 + h) - f(x_0,y_0)}{h} = M
\]
となるならば,$f$ は点 $(x_0,y_0)$ において $y$ に関して偏微分可能であるといい,
\[
\frac{\partial f}{\partial y}(x_0,y_0) := M
\]
と定義する.


\subsection{全微分}
$f : \mathbb{R}^2 \to \mathbb{R}$ を関数とし,時間依存しているとすると, 
\[
f(t)=f{x(t),y(t)}
\]
この関数を時間微分すると
\[
\frac{d f}{d t} = \frac{\partial f}{\partial x}\frac{d x}{d t}+\frac{\partial f}{\partial y}\frac{d y}{d t}
\]
ここで, この関数$f$が時間$t$ではなく任意の変数$s$に依存していたとすると, $s$での微分は
\[
\frac{d f}{d s} = \frac{\partial f}{\partial x}\frac{d x}{d s}+\frac{\partial f}{\partial y}\frac{d y}{d s}
\]
となる. 

関数が時間$t$に依存していたとしても, 任意の変数$s$に依存していたとしても, 式の形は変化しない. 
よって$dt,ds$を省略してみる. 
\[
df = \frac{\partial f}{\partial x}d x+\frac{\partial f}{\partial y}d y
\]
このように$dt,ds$を省略することを「形式的に消す」という. 
さらに, 関数がこのようにかけるとき, $f$は全微分可能であるといい, この表式を$f$の全微分という. 
突然こんなことを言われても意味が分からないと思うので, この式の意味をもう少し深堀してみる. 
そのために, まずはテイラー展開を勉強する. 

\subsection{テイラー展開}
二次未知関数を求めたい場合, その関数が通る三点の座標を知ることができれば, 連立して求めることができる. 
では, その座標を$n+1$個知るのではなく, 一つの座標自体に情報が$n+1$個あったらどうだろうか. 

$f(x) = ax^2 + bx + c$という二次関数$f$を考える. この関数を求めるには, 
値, 一階微分, 二階微分の三つの情報がわかっていればこの関数を求めることができる. 
値, 一階微分, 二階微分がそれぞれ
\[
f(1)=2, \quad f'(1)=-3 , \quad f''(1)=4
\]
のとき, 二次関数の各式を未知数 \(a,b,c\) に書き換える
\[
\begin{cases}
f(1) = a + b + c = 2,\\[2mm]
f'(1) = 2a + b = -3,\\[1mm]
f''(1) = 2a = 4.
\end{cases}
\]

二階微分の式から
\[
a = 2
\]
がわかる.

一階微分の式に代入すると
\[
2 \cdot 2 + b = 4 + b = -3 \implies b = -7.
\]

最後に関数値の式に代入すると
\[
2 + (-7) + c = -5 + c = 2 \implies c = 7.
\]

したがって,求める二次関数は
\[
\boxed{f(x) = 2x^2 - 7x + 7}
\]
となる.

より一般に, $n$次関数を求めたい場合, 値, 一階微分, ~, $n$階微分までわかっていれば, その関数を求めることができる. 
しかし, 別に三点がわかっていれば求められる関数であるし, 独立な情報が求められる部分は特に変わっていないように見える. 
この考え方の真の有難味は, 多項式関数のような関数ではなく, 
代数的に解くことができないような$f(x) = \sin x$などの超越関数を多項式(級数)で表すことができる点である!これはすごい!

では, 
\subsection{全微分とテイラー展開}


\section{ベクトル解析}


\end{document}
